%%%%%%%%%%%%%%%%%%%%%%%%%%%%%%%
%Kopf- und Fußzeilendefinition%
%%%%%%%%%%%%%%%%%%%%%%%%%%%%%%%
\usepackage[]{scrlayer-scrpage}	%ohne Trennlinien
\pagestyle{scrheadings}					%Seitenstil umdefiniert
\clearscrheadfoot					%Definition leermachen
\renewcommand{\chapterpagestyle}{scrheadings}		%Kapitelseitenstil umdefiniert
\renewcommand{\chaptermark}[1]{\markboth{\ #1}{}}	%Definition Kapitel?berschrift=\leftmark
\ohead{\textcolor{gray}{Short title up to 8 words}}	%äußere Kopfzeile
\chead{}						%mittlere Kopfzeile
\ihead{}						%innere Kopfzeile
\ofoot{\thepage}				%äußere Fu?zeile
\cfoot{}						%mittlere Fu?zeile
\ifoot{\textcolor{gray}{9th Information Management Seminar Conference, University of Zurich, Zurich 2019}}%innere Fu?zeile

%%%%%%%%%%%%%%%%%%%%%%%%%%%%%%%%%%%%%%%
%%%%%%%deutsche Zeichenkodierung%%%%%%%
%%%%%%%%%%%%%%%%%%%%%%%%%%%%%%%%%%%%%%%
\usepackage[utf8]{inputenc}
\usepackage[ngerman, english]{babel}

%%%%%%%%%%%%%%%%%%%%%%%%%%%%%%%%%%%%%%%%%%%%%%
%%%%%%%%%%%%Verschiedene Schriftarten%%%%%%%%%
%%%%%%%%%%%%%%%%%%%%%%%%%%%%%%%%%%%%%%%%%%%%%%
%\usepackage{crimson}
%\usepackage[adobe-utopia]{mathdesign}
%\usepackage{libertine}
%\usepackage{lmodern}
%\usepackage{kpfonts}
%\usepackage{stix}
%\usepackage{libertinust1math}



\usepackage[T1]{fontenc}
\fontfamily{georgia}
%\renewcommand*\familydefault{\sfdefault}


%%%%%%%%%%%%%%%%%%%%%%%%%%%%%%%%%%%%%%%%%%%%%%%%
%%%%%%%%%%%%Table of Content Formatting%%%%%%%%%
%%%%%%%%%%%%%%%%%%%%%%%%%%%%%%%%%%%%%%%%%%%%%%%%
\usepackage{tocloft}
\renewcommand{\cftpartleader}{\cftdotfill{\cftdotsep}} % for parts
\renewcommand{\cftchapleader}{\cftdotfill{\cftdotsep}} % for chapters
\renewcommand{\cftdotsep}{2} % Abstand zwischen den Punkten im Inhaltsverzeichnis
\renewcommand{\cftchapfont}{\scshape}
\renewcommand{\cftsecpagefont}{}

\RedeclareSectionCommand[beforeskip=0pt,afterskip=10pt]{chapter}
\RedeclareSectionCommand[beforeskip=10pt,afterskip=10pt]{section}
\RedeclareSectionCommand[beforeskip=10pt,afterskip=10pt]{subsection}
\setcounter{secnumdepth}{2}
\setcounter{tocdepth}{2}

%%%%%%%%%%%%%%%%%%%%%%%%%%%%%%%%%%%%%%%%%%%%%%
%%%%%%%%%%%%%%%Zusatzpackages%%%%%%%%%%%%%%%%%
%%%%%%%%%%%%%%%%%%%%%%%%%%%%%%%%%%%%%%%%%%%%%%
\usepackage{a4wide}
\usepackage{longtable}
\usepackage{longtable}
\usepackage{eurosym}					%Fancy ? Zeichen
\usepackage{acronym}					%Abk?rzungsverzeichnis
\usepackage{wasysym}					%Symbole und Sonderzeichen
\usepackage{longtable}					%Formelzeichentabelle/lange Tabelle
\usepackage{romannum}					%R?mische Zahlen
\usepackage{wrapfig}					%Bilder nebeneinander/ Text neben Bild
\usepackage{floatflt,epsfig} 				%Bilder im eps-format einf?gen
\usepackage{exceltex}					%Exceldokumente einbinden
\usepackage{multirow}					%Tabellenzellen verbinden
\usepackage{graphicx}					%ich glaube ein Vektorzeichentool
\usepackage[hyphens]{url}				%Websites einbinden
\usepackage{pdfpages} 					%direkt pdf Datein einbinden
%\usepackage{amsmath}					%matheumgebung f?r Formeln und Gleichungen
\usepackage{booktabs}					%ich glaube f?r links im PDF
\usepackage{capt-of}					%Captions f?r non-float-Objekte
\usepackage{appendix}					%erm?glicht das erstellen von Anhangsverzeichnissen
\usepackage{tikz}					%tikz Paket f?r Grafiken + Definitionen
\usetikzlibrary{arrows,positioning}
\usetikzlibrary{decorations.pathmorphing}
\usetikzlibrary{shapes.geometric}
\usepackage{xcolor}

\usepackage{chngcntr}
\counterwithout{figure}{chapter}
\counterwithout{table}{chapter}

\usepackage{caption}
\captionsetup[table]{labelsep=space, labelfont=bf}
\captionsetup[figure]{labelsep=space, labelfont=bf}

%%%%%%%%%%%%%%%%%%%%%%%%%%%%%%%%%%%%%%%%
%%%%Stil des Literaturverzeichnisses%%%%
%%%%%%%%%%%%%%%%%%%%%%%%%%%%%%%%%%%%%%%%
\bibliographystyle{abbrvdin}		%sortiert nach Autor 		- nummeriert [1]
%\bibliographystyle{unsrtdin}		%sortiert nach Textmarke	- nummeriert [1]
%\bibliographystyle{aplhadin}		%sortiert nach Autor		- [AUTOR-JAHR]


%%%%%%%%%%%%%%%%%%%%%%%%%%%%%%%%%%%%%%%%%
%%%%%%%%%%%%%%special Stuff%%%%%%%%%%%%%%
%%%%%%%%%%%%%%%%%%%%%%%%%%%%%%%%%%%%%%%%%
\usepackage[colorlinks,    				%PDF linked Verzeichnis
pdfpagelabels,
pdfstartview = FitH,
bookmarksopen = true,
bookmarksnumbered = true,
linkcolor = black,
plainpages = false,
hypertexnames = false,
citecolor = black,
breaklinks = true,
allcolors = black] {hyperref}  

\usepackage{geometry}

\geometry{
	a4paper,
	total={150mm,247mm},
	left=30mm,
	top=25mm,
	headsep=8mm,
	footskip=8mm,
}

\usepackage{setspace}
\onehalfspacing

\usepackage{newclude} % user for include
\usepackage{booktabs}
